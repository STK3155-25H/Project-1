% File: ols_mse.tex
\begin{figure}[H]
\begin{tikzpicture}
\begin{axis}[
    width=8cm,
    height=6cm,
    xlabel={Polynomial degree},
    ylabel={Error},
    title={Bias-Variance Tradeoff},
    legend pos=north east,
    grid=both,
    y tick label style={
        /pgf/number format/fixed,
        /pgf/number format/precision=2
    }
    ]
    \addplot [blue, mark=none] table [x=degree, col sep=comma] {../outputs/tables/part_g_mse_n=100.csv};
    \addlegendentry{$MSE$}

    \addplot [orange, mark=none] table [x=degree, col sep=comma] {../outputs/tables/part_g_bias2_n=100.csv};
    \addlegendentry{$Bias^2$}

    \addplot [green!70!black, mark=none] table [x=degree, col sep=comma] {../outputs/tables/part_g_variance_n=100.csv};
    \addlegendentry{$Variance$}
\end{axis}
\end{tikzpicture}
\caption{Decomposition of test MSE into bias and variance across polynomial degrees (training set size 
n
=
100
n=100). The plot illustrates the bias–variance tradeoff: low-degree models exhibit high bias and low variance, while high-degree models show low bias and increased variance, with MSE reflecting the combined effect.}
\label{fig:Bias_Var}
\end{figure}