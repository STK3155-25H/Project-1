% File: ols_mse.tex
\begin{figure}[H]
\begin{tikzpicture}
\begin{axis}[
    width=8cm,
    height=6cm,
    xlabel={Polynomial degree},
    ylabel={MSE},
    title={LASSO Regression With Different Learning Rates},
    legend pos=north east,
    grid=both,
    y tick label style={
        /pgf/number format/fixed,
        /pgf/number format/precision=2
    }
    ]
    \addplot [blue, mark=none] table [x=degree, col sep=comma] {../outputs/tables/part_e_lasso_mse_vanilla_lr=0.01.csv};
    \addlegendentry{$lr=0.01$}

    \addplot [orange, mark=none] table [x=degree, col sep=comma] {../outputs/tables/part_e_lasso_mse_vanilla_lr=0.001.csv};
    \addlegendentry{$lr=0.001$}

    \addplot [green!70!black, mark=none] table [x=degree, col sep=comma] {../outputs/tables/part_e_lasso_mse_vanilla_lr=0.0001.csv};
    \addlegendentry{$lr=0.0001$}
\end{axis}
\end{tikzpicture}
\caption{Impact of learning rate on LASSO regression test MSE using vanilla gradient descent. The figure compares model performance across polynomial degrees for three different learning rates ($0.01$, $0.001$, and $0.0001$), illustrating how step size influences convergence and generalization.}
\label{fig:GD_LASSO_LR}
\end{figure}
