% File: ols_mse.tex
\begin{figure}[H]
\begin{tikzpicture}
\begin{axis}[
    width=8cm,
    height=6cm,
    xlabel={Polynomial degree},
    ylabel={MSE},
    title={RIDGE Regression Using GD With Different Learning Rates},
    legend pos=north east,
    grid=both,
    y tick label style={
        /pgf/number format/fixed,
        /pgf/number format/precision=2
    }
    ]
    \addplot [black, mark=none, dotted] table [x=degree, col sep=comma] {../outputs/tables/part_c_mse_analytical_ridge.csv};
    \addlegendentry{$analytical$}

    \addplot [blue, mark=none] table [x=degree, col sep=comma] {../outputs/tables/part_c_mse_gd_ridge_lr=0.01.csv};
    \addlegendentry{$lr=0.01$}

    \addplot [orange, mark=none] table [x=degree, col sep=comma] {../outputs/tables/part_c_mse_gd_ridge_lr=0.001.csv};
    \addlegendentry{$lr=0.001$}

    \addplot [green!70!black, mark=none] table [x=degree, col sep=comma] {../outputs/tables/part_c_mse_gd_ridge_lr=0.0001.csv};
    \addlegendentry{$lr=0.0001$}
\end{axis}
\end{tikzpicture}
\caption{Comparison of Ridge regression test MSE across polynomial degrees for different learning rates. The figure contrasts the analytical solution with gradient descent results using step sizes $0.01$, $0.001$, and $0.0001$, showing how the choice of learning rate affects convergence toward the exact solution.}
\label{fig:GD_RIDGE_LR}
\end{figure}
