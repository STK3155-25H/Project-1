 \documentclass[
 reprint,            % two-column layout
 amsmath,amssymb,
 aps,
]{revtex4-2}



% Packages
% Packages
\usepackage[utf8]{inputenc}
\usepackage[T1]{fontenc}
\usepackage{graphicx}
\usepackage{float}
\usepackage{pgfplots}
\pgfplotsset{compat=1.18}
\usepackage{chngcntr}
\usepackage{tikz}
\usepackage{hyperref}
\usepackage{caption}
\captionsetup{hypcap=true}
\usepackage{algorithm}
\usepackage{algpseudocode}


\pgfplotsset{compat=1.18}

\begin{document}

\title{Analyzing Different Regression and Resampling Methods}

\author{Stan Daniels}
\author{Francesco Minisini}
\author{Teresa Ghirlandi}
\author{Carolina Ceccacci}
\affiliation{University of Oslo \\ Data Analysis and Machine Learning (FYS-STK3155/FYS4155)}

\date{October 6, 2025}

\begin{abstract}
Regression in machine learning is a fundamental technique for predicting outcomes based on input features. It finds relationships between variables so that predictions on unseen data can be made. 
A major challenge arises as model complexity increases:  low-degree models may underfit, while high-degree polynomial regression can become unstable and overfit the data. 
In this project, we study the Runge function, a well-known function that highlights the difficulties of high-degree polynomial interpolation. We apply Ordinary Least Squares, Ridge, and Lasso regression, complemented by gradient descent and its variants, including momentum, Adagrad, RMSprop, and ADAM. Resampling techniques such as bootstrap and cross-validation are used to evaluate model generalization and analyze the bias-variance trade-off.
The results show that OLS fits become highly unstable for high polynomial degrees, while Ridge and Lasso regularization significantly improve stability and predictive accuracy. Gradient descent methods reproduce the analytical results, though their performance depends strongly on learning-rate strategies. Overall, the study highlights the importance of regularization and resampling for controlling overfitting and improving the reliability of regression models.
\end{abstract}

\maketitle

\section{Introduction}
The aim of this project is to study various regression methods, such as Ordinary Least Squares, Ridge Regression, and Lasso Regression. It focuses on fitting polynomials to a specific one-dimensional function, the Runge function:
\[
\frac{1}{1+25x^2}
\]
The Runge function shows the difficulties of high-degree polynomial interpolation and this makes it an ideal test case to compare the performances of the different methods.
First, an OLS regression analysis is performed, exploring the dependence on the number of data points and the degree of polynomial. The analysis is then extended to Ridge and Lasso regressions, which add a regularization parameter ${\lambda}$. Gradient descent methods are implemented. The analysis starts with the standard gradient descent method, but then, to improve efficiency and convergence, several variants of the gradient descent have been developed, such as momentum, stochastic gradient descent and adaptive methods, including Adagrad, RMSprop, ADAM. The performance of OLS, Ridge and Lasso is then compared with the gradient descent-based optimization methods. In order to evaluate model performance and investigate bias-variance trade-off, resampling techniques such as bootstrap and cross-validation are applied, highlighting how different choices of model complexity and regularization affect the trade-off between bias and variance. These techniques provide insight into the stability of the models and the reliability of their predictions. 
Overall, this project aims to illustrate the strengths and the limitations of each method.\\
The structure of this project is as follows: 
\begin{itemize}
    \item Section II "Methods", describes the regression techniques and optimization algorithms, as well as the resampling methods.
    \item Section III "Results and Discussion", presents the numerical results, compares the performance of the different methods and discusses their implications in terms of bias-variance trade-off
    \item section IV "Conclusion", summarizes the main results and the insights gained from the methods studied.
\end{itemize}


\section{Methods}
 Let $\mathbf{y} \in \mathbb{R}^n$ denote the vector of target values and $\mathbf{X} \in \mathbb{R}^{n \times p}$ the design matrix containing $p$ predictors for $n$ observations. The following linear model is assumed:
$$
\mathbf{y} = \tilde{\mathbf{y}} + \boldsymbol{\epsilon}, \quad \text{with} \quad \tilde{\mathbf{y}} = \mathbf{X}\boldsymbol{\theta},
$$

where $\tilde{\mathbf{y}}$ represents the predictions of the model, $\boldsymbol{\theta} \in \mathbb{R}^p$ is the vector of unknown coefficients to be estimated and $\boldsymbol{\epsilon}$ is a vector of errors, typically assumed to be independent and identically distributed with zero mean and variance $\sigma^2$.

The goal of regression is to find an estimate of the optimal parameter $\boldsymbol{\theta}$ that best explains the observed data according to a chosen criterion. 

A detailed description of the methods follows.

\subsection{Scaling Of The Data}
Before applying any regression and optimization methods, the dataset was split into a training and testing dataset and scaled appropriately.
This was performed using the \texttt{split\_scale} and \texttt{polynomial\_features\_scaled} functions.
First the training and test sets were seperated.\\
The input feature $x$ was then standardized using statistics computed on the training set: each feature column was centered by subtracting its mean and scaled by dividing by its standard deviation.
This process produces features with zero mean and unit variance.
The same mean and standard deviation from the training set were applied to scale the test set, avoiding any data leakage.\\
For polynomial regression, each feature was expanded into polynomial terms up to a maximum degree.
After generating the polynomial features, each column (except the intercept) was scaled using the mean and standard deviation of the corresponding column from the training set, via the \texttt{polynomial\_features\_scaled} function.
This ensures that all polynomial features are on a comparable scale across training and test sets.\\\\
Scaling was necessary for several reasons.\\
First, high-degree polynomial terms can grow very large, causing numerical instability in the design matrix.
Second, regularization methods like Ridge and Lasso assume features are on a comparable scale, otherwise the penalty distorts model behavior.
Third, gradient-based optimizers converge faster and more stably when the features are standardized.
Finally, applying training-set statistics to scale the test set ensures unbiased evaluation.


\subsection{Ordinary Least Squares}
Ordinary Least Squares (OLS) is the classical method for linear regression and it estimates $\boldsymbol{\theta}$ by minimizing the mean squared error\cite{hjorthjensen2025week35}. This is the cost function that is going to be optimize:

$$
C(\boldsymbol{\theta})=\frac{1}{n}\left\{\left(\boldsymbol{y}-\boldsymbol{X}\boldsymbol{\theta}\right)^T\left(\boldsymbol{y}-\boldsymbol{X}\boldsymbol{\theta}\right)\right\}$$

It means that it's required that the derivative with respect to $\boldsymbol{\theta}$ be set equal to zero: $$\frac{\partial C(\boldsymbol{\theta)}}{\partial \boldsymbol{\theta_j}} =0= \mathbf{X}^\top(\mathbf{y-X\boldsymbol{\theta})}$$

$$\mathbf{X}^\top \mathbf{y}= \mathbf{X}^\top \mathbf{X} \boldsymbol{\theta}$$

For a full-rank design matrix, this has the closed-form solution:
$$
\hat{\boldsymbol{\theta}}_{\text{OLS}} = (\mathbf{X}^\top \mathbf{X})^{-1} \mathbf{X}^\top \mathbf{y}
$$

OLS provides a simple solution, suitable when the features are few and not highly correlated.

\subsubsection{Implementation}
The inputs of the OLS function are a feature matrix $\mathbf{X} \in \mathbb{R}^{n \times p}$ and a vector of targets $\mathbf{y} \in \mathbb{R}^n$. The output is the vector of the parameters $\boldsymbol{\theta} \in \mathbb{R}^{p}$, whose fomula is given above.\\
The function is implemented in Python, using the NumPy library for efficient numerical computations. The function \texttt{np.linalg.pinv} is used to compute the pseudoinverse of the matrix $\mathbf{X}^\top \mathbf{X}$, which is particularly useful when $\mathbf{X}^\top \mathbf{X}$ is not invertible. This ensures numerical stability.

\subsection{Ridge regression}
A regularization parameter $\lambda$ can be introduced by defining a new cost function to be optimized, that is:
$$ C(\boldsymbol{\theta}) 
=\frac{1}{n}\vert\vert \boldsymbol{y}-\boldsymbol{X}\boldsymbol{\theta}\vert\vert_2^2+\lambda\vert\vert \boldsymbol{\theta}\vert\vert_2^2
$$
where the second term represents an $L^2$ penalty on the size of the coefficients.
This leads to the Ridge regression minimization problem where it is required that $ |\boldsymbol{\theta}\|_2^2 \leq t$, where $t$ is a finite positive number. 
One of the main motivations behind Ridge is its ability to resolve the problem of non-invertibility of $\mathbf{X}^\top \mathbf{X}$, which often arises when features are highly correlated. Ridge regression resolves this problem by adding the parameter $\boldsymbol{\lambda}$ to the diagonal of $\mathbf{X}^\top \mathbf{X}$ before inverting it.
Taking the derivatives with respect to $\boldsymbol{\theta}$ the optimal parameters are obtained:
$$ \boldsymbol{\hat\theta_{Ridge}}=(\mathbf{X}^\top \mathbf{X} + \boldsymbol{\lambda}\mathbf{I})^{-1} \mathbf{X}^\top \mathbf{y}$$
with $\mathbf{I}$ being a $p\times p$ identity matrix.

\subsubsection{Implementation}
The inputs of the Ridge function are a feature matrix $\mathbf{X} \in \mathbb{R}^{n \times p}$, a vector of targets $\mathbf{y} \in \mathbb{R}^n$, the regularization parameter $\lambda$ and the intercept. The output is the vector of the parameters $\boldsymbol{\theta} \in \mathbb{R}^{p}$, whose fomula is given above.\\
The function uses the same function \texttt{np.linalg.pinv} to compute the pseudoinverse of the matrix $\mathbf{X}^\top \mathbf{X} + \boldsymbol{\lambda}\mathbf{I}$, ensuring numerical stability.

\subsection{Lasso regression}
Here the regularization term is based on the $L_1$ norm of the parameters. The cost function is defined as
$$
C(\boldsymbol{\theta}) 
= \frac{1}{n}\vert\vert \boldsymbol{y}-\boldsymbol{X}\boldsymbol{\theta}\vert\vert_2^2
+ \lambda \, \|\boldsymbol{\theta}\|_1,
$$
where $\|\boldsymbol{\theta}\|_1 = \sum_{j}|\theta_j|$.  
This formulation leads to the Lasso minimization problem where it is required that $\|\boldsymbol{\theta}\|_1 \leq t$, with $t$ being a finite positive number. 

Unlike Ridge regression, taking the derivatives with respect to $\boldsymbol{\theta}$ does not lead to an analytical solution. 
The optimization problem can however be solved by using iterative gradient descent methods.\\ \\
The key feature of Lasso lies in its ability to shrink some estimated coefficients $\hat{\theta}_j$ exactly to zero. 
When this happens, the corresponding predictor is completely removed from the model.  
In contrast, Ridge regression never eliminates variables: it only shrinks the coefficients $\hat{\theta}_j$ towards zero but keeps all predictors in the model.
Typically, Lasso Regression is preferred when the goal is to simplify the model and improve interpretability, especially when there are a lot of features. On the other hand, Ridge regression is better for handling multicollinearity among features. 


\subsection{Gradient descent}

Although OLS and Ridge regression have analytical solutions, such solutions are not always available in general, so a numerical approach is often needed to optimize the same cost function.\\

Consider the cost function $C(\boldsymbol{\theta})$ that has to be minimized with respect to the parameters $\boldsymbol{\theta}$.  
A second-order Taylor expansion around a point $\boldsymbol{\theta}_n$ is performed:
$$
C(\boldsymbol{\theta}) \approx C(\boldsymbol{\theta}_n) + (\boldsymbol{\theta}-\boldsymbol{\theta}_n)^\top \nabla_{\boldsymbol{\theta}} C(\boldsymbol{\theta}_n) 
+ \frac{1}{2} (\boldsymbol{\theta}-\boldsymbol{\theta}_n)^\top \mathbf{H}(\boldsymbol{\theta}_n) (\boldsymbol{\theta}-\boldsymbol{\theta}_n),
$$

where $\mathbf{H}(\boldsymbol{\theta}_n)$ is the Hessian matrix of second derivatives at $\boldsymbol{\theta}_n$.

Neglecting the second-order term (or assuming it is costly to compute), a first-order approximation gives the update rule in the direction of the steepest descent:
\[
\boldsymbol{\theta}_{n+1} = \boldsymbol{\theta}_n - \eta \, \nabla_{\boldsymbol{\theta}} C(\boldsymbol{\theta}_n),
\]
where $\eta > 0$ is a learning rate controlling the step size.  

This iterative procedure moves the parameters towards the minimum of $C(\boldsymbol{\theta})$. For convex functions such as the mean squared error in linear regression, convergence is guaranteed if $\eta$ is chosen appropriately.

\subsubsection{Implementation}
The iteration start from an initial guess $\boldsymbol{\theta}^{(0)}$, the parameters are updated iteratively according to the rule:
\[
\boldsymbol{\theta}^{(n+1)} = \boldsymbol{\theta}^{(n)} - \eta \, \nabla_{\boldsymbol{\theta}} C(\boldsymbol{\theta}^{(n)}),
\]
where $\eta > 0$ is the learning rate and $\nabla_{\boldsymbol{\theta}} C(\boldsymbol{\theta}^{(n)})$ is the gradient of the cost function with respect to $\boldsymbol{\theta}$ at iteration $t$.
After each update, the algorithm recomputes the cost function and its gradient until a convergence criterion is met.\\

\begin{algorithm}[H]
\caption{Gradient Descent}
\begin{algorithmic}[1]
\State Initialize $\boldsymbol{\theta}^{(0)}$
\For{$n = 0, 1, 2, \dots$ until convergence}
    \State Compute gradient $\nabla_{\boldsymbol{\theta}} C(\boldsymbol{\theta}^{(n)})$
    \State Update parameters: 
    $\boldsymbol{\theta}^{(n+1)} = \boldsymbol{\theta}^{(n)} - \eta \, \nabla_{\boldsymbol{\theta}} C(\boldsymbol{\theta}^{(n)})$
\EndFor
\end{algorithmic}
\end{algorithm}

(AAA: I don't know if we want something like this)

\subsection{Stochastic Gradient Descent (SGD)}
Gradient descent can be computationally expensive when applied to large datasets. Stochastic Gradient Descent (SGD) addresses this by updating the model parameters using a single (or small batch of) data point(s) per iteration, rather than the full dataset. This introduces noise in the updates but significantly reduces computation time. The update rule is as follows:  
$$
\boldsymbol{\theta}_{n+1} = \boldsymbol{\theta}_n - {\eta \boldsymbol{\nabla}} l_{i_n} (\boldsymbol{\theta_t}),
$$  
where \(i_t\) is a randomly selected data point and \(\eta\) is the learning rate. (AAA: cite lecture)\\

While Stochastic Gradient Descent is faster and more memory-efficient than full-batch gradient descent, it introduces noise in the parameter updates, which can make convergence less stable and less precise. However, this stochasticity can be beneficial, as it can help escape poor local minima and explore the optimization landscape more effectively, making it well suited for large datasets.

\subsubsection{Implementation}
\begin{algorithm}[H]
\caption{Stochastic Gradient Descent}
\begin{algorithmic}[1]
\State Initialize $\boldsymbol{\theta}^{(0)}$
\For{$n = 0, 1, 2, \dots$ until convergence}
    \State Randomly select a data point (or mini-batch) $i_t$
    \State Compute stochastic gradient: $\nabla_{\boldsymbol{\theta}} \ell_{i_n}(\boldsymbol{\theta}^{(n)})$
    \State Update: $\boldsymbol{\theta}^{(n+1)} \gets \boldsymbol{\theta}^{(n)} - \eta \, \nabla_{\boldsymbol{\theta}} \ell_{i_n}(\boldsymbol{\theta}^{(n)})$
\EndFor
\end{algorithmic}
\end{algorithm}

\subsection{Momentum}
The learning rate $\eta$ plays a crucial role in the convergence and a limitation of these methods is the fixed learning rate ${\eta}$: 
\begin{itemize}
    \item if $\eta$ is too large, the updates can overshoot the minimum, causing oscillations or divergence
    \item if $\eta$ is too small, convergence is very slow
\end{itemize}
Moreover, for a function with steep directions and flat directions, a single global $ \eta$ may be inappropriate:
steep coordinates require a smaller step size to avoid oscillation and flat coordinates could use a larger step to speed up progress.

In order to mitigate this problem, gradient descent with momentum is introduced: it refers to a method that smoothens the optimization trajectory by adding a term that helps the optimizer remember the past gradients.

Mathematically, let $\boldsymbol{v}_n$ denote the velocity (or accumulated gradient) at iteration $n$. The update rules for gradient descent with momentum are:
$$
\boldsymbol{v}_{n+1} = \gamma \, \boldsymbol{v}_n - \eta \, \nabla_{\boldsymbol{\theta}} C(\boldsymbol{\theta}_n),
$$
$$
\boldsymbol{\theta}_{n+1} = \boldsymbol{\theta}_n + \boldsymbol{v}_{n+1},
$$
where $\gamma \in [0,1)$ is the momentum coefficient, which controls how much of the past gradients are remembered in the current update. A value close to 1 means the optimizer will have more inertia while a value closer to 0 means less reliance on past gradients. This mechanism enables the algorithm to suppress oscillations along steep directions while simultaneously accelerating progress across flatter regions of the cost surface. The result is a convergence process that is both faster and more stable than standard gradient descent.\\
This leads to more advanced optimization methods which use an adaptive learning rate for each parameter that depends on the history of gradients.\\


\subsection{Adagrad}
Adagrad adapts the learning rate for each parameter based on the historical squared gradients, giving smaller updates to frequently updated parameters and larger updates to infrequent ones. This is particularly useful for sparse data.  
\[
\theta_{t+1} = \theta_t - \frac{\eta}{\sqrt{G_t} + \epsilon} \, \nabla l_{i_t}(\theta_t),
\]

where \(G_t\) is the sum of squared gradients up to time \(t\) and \(\epsilon\) prevents division by zero.  \\

The accumulation of squared gradients can lead to excessively small learning rates over time, slowing down convergence.

\subsubsection{Implementation}
(AAA)

\subsection{RMSprop}
RMSprop tackles Adagrad's excessive shrinkage of the learning rate by using an exponentially decaying average of past squared gradients instead of a simple sum. This prevents the learning rate from shrinking too much and slowing down.  
If the running average is
\[
v_t = \rho v_{t-1} + (1 - \rho)\left( \nabla C(\theta_t) \right)^2,
\]
with decay rate \(\rho\) 0.9 or 0.99,

the new parameter update is
\[\theta_{t+1} = \theta_t - \frac{\eta}{\sqrt{v_t + \epsilon}} \nabla C(\theta_t).
\]
(AAA cite lecture notes)
On the other hand, RMSprop requires tuning of the decay rate \(\rho\) and the base learning rate \(\eta\).

\subsubsection{Implementation}
(AAA)

\subsection{Adam}
Adam (Adaptive Moment Estimation) combines momentum and RMSprop, maintaining both an exponentially decaying average of past gradients (first moment) and squared gradients (second moment), with bias correction. It adapts the learning rate per parameter, providing stability, while incorporating momentum, accelerating the convergence.
So, combining the moving averages of momentum
\[
m_t = \beta_1 m_{t-1} + (1-\beta_1) \nabla_\theta C(\theta_t)
\]  
and of the squared gradients
\[
v_t = \beta_2 v_{t-1} + (1-\beta_2) (\nabla_\theta C(\theta_t))^2,
\]  
and correcting bias
\[
\hat{m}_t = \frac{m_t}{1-\beta_1^t}, \quad \hat{v}_t = \frac{v_t}{1-\beta_2^t},
\]  
the update rule in Adam becomes
\[
\theta_{t+1} = \theta_t - \frac{\eta}{\sqrt{\hat{v}_t}+\epsilon} \hat{m}_t,
\]  
where \(\epsilon\) prevents division by zero.

Adam can sometimes converge to slightly worse minima than SGD in certain tasks, and hyperparameter tuning is still required.
\subsubsection{Implementation}
(AAA)

\subsection{Gradient descents comparison}
Stochastic Gradient Descent (SGD) is appreciated for its simplicity and memory efficiency, particularly when tackling large datasets. However, it sometimes oscillates and may converge slowly. 

Momentum builds upon SGD by accelerating convergence in directions of consistent descent, although it can occasionally bypass the lowest point in regions where the cost function curves steeply. 

Adagrad's strength lies in its adaptability to sparse data, but it tends to reduce the learning rate over time, which can stall progress. 

RMSprop was introduced to counter Adagrad's diminishing learning rate, making it effective for online and non-stationary tasks where data characteristics frequently change. 

Adam is more often the preferred method because it merges the advantages of momentum and adaptive learning rates, generally performing robustly in most deep learning scenarios, though it may require careful parameter tuning to reach optimal performance.

(AAA in our case)

\subsection{Bias-variance trade-off and resampling techniques}

The bias-variance trade-off is a fundamental concept in machine learning and statistics that describes the relationship between a model's complexity, the accuracy of its predictions, and how well it can make predictions on previously unseen data that were not used to train the model. 
Bias is the error due to approximating a complex real-world problem with a simplified model: 
$$\mathrm{Bias}[\tilde y]= \mathbb{E}[(f-\mathbb{E}[\tilde y])^2]$$

It's the squared difference between the true function f(x) and the average prediction $\mathbb{E}[\tilde y]$ of the model over different training sets. High bias leads to underfitting, where the model is too simple (often linear) and cannot capture the underlying patterns, resulting in large errors on both training and test data.
\\Variance measures the sensitivity of a model to small changes in the training data:
$$
\mathrm{Var}[\tilde y] = \mathbb{E}[\tilde y^2] - (\mathbb{E}[\tilde y])^2
$$
It's the expected squared deviation of the model's predictions from its average prediction.
High variance leads to overfitting, where the model fits the training data very closely but performs poorly on unseen data, showing large test errors.\\\\
More formally, consider a dataset $\{(x_i, y_i)\}_{i=1}^n$, where the true data is generated by a noisy model $y = f(x) + \varepsilon$ and $\varepsilon$ is the noise term with zero mean and variance $\sigma^2$. 
The true function $f(x)$ is approximated by a model $\tilde y$, which depends on parameters $\boldsymbol{\theta}$ and a design matrix $X$, determined by minimizing the mean squared error (MSE):
$$
\text{C}(\boldsymbol{X, \theta}) = \frac{1}{n} \sum_{i=1}^n \big( y_i - \tilde y \big)^2= \mathbb{E}\big[(y - \tilde y)^2\big]
$$

The expected prediction error can be decomposed as
$$\mathbb{E}[(y - \tilde y)^2\big] = \mathrm{Bias}[\tilde y] + \mathrm{Var}[\tilde y] + \sigma^2$$
where $\sigma ^2$ is the variance of the errore $\epsilon$. 

To derive this equation, it should be recalled that the variance of $\boldsymbol{y}$ and $\boldsymbol{\epsilon}$ are both equal to $\sigma^2$. The mean value of $\boldsymbol{\epsilon}$ is by definition equal to zero. Furthermore, the function $f$ is not a stochastics variable, idem for $\boldsymbol{\tilde{y}}$.
Using a more compact notation in terms of the expectation value
$$
\mathbb{E}\left[(\boldsymbol{y}-\boldsymbol{\tilde{y}})^2\right]=\mathbb{E}\left[(\boldsymbol{f}+\boldsymbol{\epsilon}-\boldsymbol{\tilde{y}})^2\right],
$$

and adding and subtracting $\mathbb{E}\left[\boldsymbol{\tilde{y}}\right]$ one obtains 
$$
\mathbb{E}\left[(\boldsymbol{y}-\boldsymbol{\tilde{y}})^2\right]=\mathbb{E}\left[(\boldsymbol{f}+\boldsymbol{\epsilon}-\boldsymbol{\tilde{y}}+\mathbb{E}\left[\boldsymbol{\tilde{y}}\right]-\mathbb{E}\left[\boldsymbol{\tilde{y}}\right])^2\right],
$$

which, using the abovementioned expectation values can be rewritten as
$$
\mathbb{E}\left[(\boldsymbol{y}-\boldsymbol{\tilde{y}})^2\right]=\mathbb{E}\left[(\boldsymbol{y}-\mathbb{E}\left[\boldsymbol{\tilde{y}}\right])^2\right]+\mathrm{Var}\left[\boldsymbol{\tilde{y}}\right]+\sigma^2,
$$

that is the rewriting in terms of the bias, the variance of the model $\boldsymbol{\tilde{y}}$ and the variance of $\boldsymbol{\epsilon}$. \cite{misc} 
In order to derive this equation, the assumption that is made is that the unknown function $\boldsymbol{f}$ can be replaced by the target data $\boldsymbol{y}$.\\
*I'm not able to cite because the .bib is deleting what I'm writing* \\\\ 
This decomposition underlies the bias-variance trade-off: a high bias and low variance situation corresponds to a model that is too simple and underfits the data, while a low bias and high variance situation corresponds to a model that is too complex and overfits the data.
The goal is to find a balance, achieving a model that is complex enough to capture the underlying patterns but simple enough to generalize well to new data.
Resampling techniques provide the practical tool for measuring and managing this balance, by reliably estimating the generalization ability of the model.
They involve repeatedly drawing samples from a training set and refitting a model of interest on each sample in order to obtain additional information about the fitted model. Two resampling techniques are discussed here: bootstrap and cross-validation.\\\\
Bootstrap is a resampling technique that involves repeatedly drawing samples with replacement from the original dataset to create multiple "bootstrap" samples. Each bootstrap sample is used to fit the model, and the variability of the model's predictions across these samples provides an estimate of the model's variance. This method is particularly useful for estimating the uncertainty of model parameters and assessing the stability of the model.\\\\
Cross-validation is a technique used to assess how the results of a statistical analysis will generalize to an independent dataset. The most common form is k-fold cross-validation, where the dataset is divided into k subsets (or "folds"). The model is trained on k-1 folds and validated on the remaining fold. This process is repeated k times, with each fold serving as the validation set once. The average performance across all k trials provides a robust estimate of the model's generalization error. Cross-validation helps in selecting model parameters and in preventing overfitting by ensuring that the model performs well on unseen data.\\\\

\section{Results and Discussion}
The performance of the regression models are evaluated using both the MSE and the score $R^2$. The definitions of these are the following:
$$ \mathrm{MSE} = \frac{1}{n} \sum_{i=1}^{n} \left( y_i - \tilde{y}_i \right)^2$$
$$R^2 = 1 - \frac{\sum_{i=1}^{n} \left( y_i - \tilde{y}_i \right)^2}{\sum_{i=1}^{n} \left( y_i - \bar{y} \right)^2}$$
where $n$ is the number of data points, $y_i$ are the true values, $\tilde{y}_i$ are the predicted values, and $\bar{y}$ denotes the mean of the true values.

\subsection{OLS}

The analysis starts with performing a standard ordinary least squares regression using polynomials in $x$ up to order 15. The dependence of the polynomial degree and number of data point has been explored by plotting the mse as a function of polynomial degree for different amount of data points. This can be seen in Figure \ref{fig:OLS_graph}.\\\\
As one can see, generally the mse goes down as the polynomial degree increases. But for a small amount of data points like $n = 40$ and $n = 50$, when the polynomial degree goes to high the mse starts to increase again.\\
This happens because of overfitting. When the polynomial degree increases the model tries to fit the data too perfectly and will fit the underlying random noise as well and the model will then perform poorly of unseen data. For a high number of data points this is barely noticeable but for a lower amount of data points the effect will become more pronounced.\\\\
$R^2$ ?\\\\
When plotting the parameters $\theta$ as a function of polynomial degree, it can be observed that for low-degree polynomials the coefficients remain relatively stable, whereas for higher-degree polynomials they increase rapidly in magnitude and exhibit pronounced oscillations.
This behavior reflects overfitting: at high degrees, the model captures both the true signal and the noise, causing the coefficients to become unstable. Using more data points reduces this effect and stabilizes the parameter estimates
    
% File: ols_mse.tex
\begin{tikzpicture}
\begin{axis}[
    width=8cm,
    height=6cm,
    xlabel={Polynomial degree},
    ylabel={MSE (Test)},
    title={Test MSE vs Polynomial degree},
    legend pos=north east,
    grid=both
    ]
    \addplot [red, mark=none] table [x=degree, y={n=40}, col sep=comma] {../outputs/tables/part_a_mse_test.csv};
    \addlegendentry{$n=40$}

    \addplot [blue, mark=none] table [x=degree, y={n=50}, col sep=comma] {../outputs/tables/part_a_mse_test.csv};
    \addlegendentry{$n=50$}

    \addplot [green!70!black, mark=none] table [x=degree, y={n=100}, col sep=comma] {../outputs/tables/part_a_mse_test.csv};
    \addlegendentry{$n=100$}

    \addplot [orange, mark=none] table [x=degree, y={n=500}, col sep=comma] {../outputs/tables/part_a_mse_test.csv};
    \addlegendentry{$n=500$}

    \addplot [purple, mark=none] table [x=degree, y={n=1000}, col sep=comma] {../outputs/tables/part_a_mse_test.csv};
    \addlegendentry{$n=1000$}
\end{axis}
\end{tikzpicture}


\subsection{Ridge}
\textbf{We can also put the theta values graph in and show that the ridge method lowers the extreme high theta values for high degrees.}\\\\
Now, Ridge regression is applied to the same dataset, exploring the impact of different regularization parameters $\lambda$. First, the effect of the regularization parameter $\lambda $ on the MSE as a function of polynomial degree was analyzed for a fixed number of data points n=100: the results are shown in Figure \ref{fig:RIDGE_MSE_Lambda}.
Small values of $\lambda$ (e.g., 0.001, 0.01) behave similarly to OLS, with the MSE decreasing as polynomial degree increases. However, as $\lambda$ increases (e.g., 1, 10, 100), the MSE curve flattens and even increases for high polynomial degrees, indicating that strong regularization prevents overfitting but may lead to underfitting if too large.
Among the tested values, $\lambda$ =0.01 produces the lowest MSE for most polynomial degrees, suggesting an optimal trade-off between bias and variance.

% File: ols_mse.tex
\begin{figure}[H]
\begin{tikzpicture}
\begin{axis}[
    width=8cm,
    height=6cm,
    xlabel={Polynomial degree},
    ylabel={MSE},
    title={RIDGE Regression For Different $\lambda$ Values},
    legend pos=north east,
    legend style= {font = \tiny},
    grid=both,
    y tick label style={
        /pgf/number format/fixed,
        /pgf/number format/precision=2
    },
    ymax=0.2,
    restrict y to domain=0:0.2
    ]
    \addplot [blue, mark=none] table [x=degree, y={lambda=0}, col sep=comma] {../outputs/tables/part_b_mse_n100_all_lambdas.csv};
    \addlegendentry{$\lambda=0$}

    \addplot [orange, mark=none] table [x=degree, y={lambda=0.001}, col sep=comma] {../outputs/tables/part_b_mse_n100_all_lambdas.csv};
    \addlegendentry{$\lambda=0.001$}

    \addplot [green!70!black, mark=none] table [x=degree, y={lambda=0.01}, col sep=comma] {../outputs/tables/part_b_mse_n100_all_lambdas.csv};
    \addlegendentry{$\lambda=0.01$}

    \addplot [red, mark=none] table [x=degree, y={lambda=0.1}, col sep=comma] {../outputs/tables/part_b_mse_n100_all_lambdas.csv};
    \addlegendentry{$\lambda=0.1$}

    \addplot [purple, mark=none] table [x=degree, y={lambda=1}, col sep=comma] {../outputs/tables/part_b_mse_n100_all_lambdas.csv};
    \addlegendentry{$\lambda=1$}

    \addplot [brown, mark=none] table [x=degree, y={lambda=10}, col sep=comma] {../outputs/tables/part_b_mse_n100_all_lambdas.csv};
    \addlegendentry{$\lambda=10$}

    \addplot [magenta, mark=none] table [x=degree, y={lambda=100}, col sep=comma] {../outputs/tables/part_b_mse_n100_all_lambdas.csv};
    \addlegendentry{$\lambda=100$}
\end{axis}
\end{tikzpicture}
\caption{Test MSE vs. polynomial degree for Ridge regression with varying regularization parameter $\lambda$ ($n=100$).}
\label{fig:RIDGE_MSE_Lambda}
\end{figure}


Next, $\lambda = 0.01$ is selected to plot the MSE as a function of polynomial degree for different numbers of data points, allowing a direct comparison with the standard OLS regression, as shown in Figure \ref{fig:RIDGE_MSE_data}. 
The comparison reveals that while both methods achieve low MSE at moderate polynomial degrees, OLS exhibits a sharp increase in MSE at high degrees due to overfitting. In contrast, Ridge regression maintains a lower and more stable MSE across all degrees, demonstrating its effectiveness in controlling model complexity through regularization. This highlights Ridge's advantage in scenarios where overfitting is a concern, particularly with high-degree polynomials.
As one can see, the mse decreases as the polynomial degree increases, similar to OLS.
% File: ols_mse.tex
\begin{figure}[H]
\begin{tikzpicture}
\begin{axis}[
    width=8cm,
    height=6cm,
    xlabel={Polynomial degree},
    ylabel={MSE},
    title={RIDGE Regression For Different Amount Of Data Points},
    legend pos=north east,
    grid=both,
    y tick label style={
        /pgf/number format/fixed,
        /pgf/number format/precision=2
    }
    ]
    \addplot [blue, mark=none] table [x=degree, y={n=40}, col sep=comma] {../outputs/tables/part_b_mse_test_lambda=0.01.csv};
    \addlegendentry{$n=40$}

    \addplot [orange, mark=none] table [x=degree, y={n=50}, col sep=comma] {../outputs/tables/part_b_mse_test_lambda=0.01.csv};
    \addlegendentry{$n=50$}

    \addplot [green!70!black, mark=none] table [x=degree, y={n=100}, col sep=comma] {../outputs/tables/part_b_mse_test_lambda=0.01.csv};
    \addlegendentry{$n=100$}

    \addplot [red, mark=none] table [x=degree, y={n=500}, col sep=comma] {../outputs/tables/part_b_mse_test_lambda=0.01.csv};
    \addlegendentry{$n=500$}

    \addplot [purple, mark=none] table [x=degree, y={n=1000}, col sep=comma] {../outputs/tables/part_b_mse_test_lambda=0.01.csv};
    \addlegendentry{$n=1000$}
\end{axis}
\end{tikzpicture}
\caption{Test mean squared error (MSE) as a function of polynomial degree for different numbers of data points $n$. The regularization parameter is fixed at $\lambda = 0.01$.}
\label{fig:RIDGE_MSE_data}
\end{figure}


The parameters $\theta$ as a function of polynomial degree for Ridge regression are plotted. Compared to OLS, the coefficients remain more stable even at high polynomial degrees. 
This stability indicates that Ridge's $L_2$ regularization effectively constrains the magnitude of the coefficients, preventing them from becoming excessively large and reducing overfitting.
The result is a more robust model that generalizes better to unseen data.

\subsection{Gradient Descent Methods}

The analytical expression for paramters $\theta$ is now compared to the results obtained using gradient descent methods.\\\\
First, the standard gradient descent method is applied to both OLS and Ridge regression, exploring the impact of different learning rates $\eta$. 
The results for OLS and Ridge with varying learning rates are shown in the figures below and it is evident that the gradient descent is very sensitive to the choice of the learning rate.\\
For example, with a learning rate equale to 0.01, as shown in the Figure \ref{fig:GD_OLS_LR}, the method converges to the analytical solution for polynomial degrees up to 15, with the MSE closely matching the analytical results.
For smaller learning rates, the algorithm does not reach the analytical solution within the set number of iterations, resulting in a higher MSE. 
For learning rate bigger than 0.01, the algorithm diverges, failing to converge to the analytical solution and causing the MSE to increase dramatically.
% File: ols_mse.tex
\begin{figure}[H]
\begin{tikzpicture}
\begin{axis}[
    width=8cm,
    height=6cm,
    xlabel={Polynomial degree},
    ylabel={MSE},
    title={OLS Using GD With Different Learning Rates},
    legend pos=north east,
    grid=both,
    legend style= {font = \tiny},
    y tick label style={
        /pgf/number format/fixed,
        /pgf/number format/precision=2
    }
    ]
    \addplot [black, mark=none, dotted] table [x=degree, col sep=comma] {../outputs/tables/part_c_mse_analytical_ols.csv};
    \addlegendentry{$analytical$}

    \addplot [blue, mark=none] table [x=degree, col sep=comma] {../outputs/tables/part_c_mse_gd_ols_lr=0.01.csv};
    \addlegendentry{$lr=0.01$}

    \addplot [orange, mark=none] table [x=degree, col sep=comma] {../outputs/tables/part_c_mse_gd_ols_lr=0.001.csv};
    \addlegendentry{$lr=0.001$}

    \addplot [green!70!black, mark=none] table [x=degree, col sep=comma] {../outputs/tables/part_c_mse_gd_ols_lr=0.0001.csv};
    \addlegendentry{$lr=0.0001$}
\end{axis}
\end{tikzpicture}
\caption{Test MSE of OLS regression vs. polynomial degree for the analytical solution and gradient descent with step sizes $\eta=0.01$, $0.001$, and $0.0001$, showing the effect of learning rate on convergence to the analytical optimum.}
\label{fig:GD_OLS_LR}
\end{figure}

Similarly, for Ridge regression with $\lambda = 0.01$, as shown in the Figure \ref{fig:GD_RIDGE_LR}, a learning rate of 0.01 allows convergence to the analytical solution, while smaller rates lead to slower convergence and larger MSE.
% File: ols_mse.tex
\begin{figure}[H]
\begin{tikzpicture}
\begin{axis}[
    width=8cm,
    height=6cm,
    xlabel={Polynomial degree},
    ylabel={MSE (Test)},
    title={RIDGE MSE vs Polynomial degree(lambda = 0.01)},
    legend pos=north east,
    grid=both,
    y tick label style={
        /pgf/number format/fixed,
        /pgf/number format/precision=2
    }
    ]
    \addplot [black, mark=none, dotted] table [x=degree, col sep=comma] {../outputs/tables/part_c_mse_analytical_ridge.csv};
    \addlegendentry{$analytical$}

    \addplot [green!70!black, mark=none] table [x=degree, col sep=comma] {../outputs/tables/part_c_mse_gd_ridge_lr=0.01.csv};
    \addlegendentry{$lr=0.01$}

    \addplot [orange, mark=none] table [x=degree, col sep=comma] {../outputs/tables/part_c_mse_gd_ridge_lr=0.001.csv};
    \addlegendentry{$lr=0.001$}

    \addplot [purple, mark=none] table [x=degree, col sep=comma] {../outputs/tables/part_c_mse_gd_ridge_lr=0.0001.csv};
    \addlegendentry{$lr=0.0001$}
\end{axis}
\end{tikzpicture}
\caption{\textbf{caption here}}
\label{fig:GD_RIDGE_LR}
\end{figure}

 
In both cases, the analytical solution is more accurate than gradient descent, except for a learning rate of 0.01, where the two methods yield nearly identical results.\\\\
Next, advanced gradient descent methods are applied to both OLS and Ridge regression.
For OLS, as shown in the Figure \ref{fig:AGD_OLS}, Momentum, Adagrad, RMSprop, and Adam all successfully converge to the analytical solution with MSE values closely matching the analytical results across polynomial degrees.
 RMSprop and Adam show better performance at higher polynomial degrees, because of their adaptive learning rate mechanisms.

% File: ols_mse.tex
\begin{figure}[H]
\begin{tikzpicture}
\begin{axis}[
    width=8cm,
    height=6cm,
    xlabel={Polynomial degree},
    ylabel={MSE},
    title={OLS With Advanced GD Methods},
    legend pos=north east,
    grid=both,
    y tick label style={
        /pgf/number format/fixed,
        /pgf/number format/precision=2
    }
    ]
    \addplot [black, mark=none, dotted] table [x=degree, col sep=comma] {../outputs/tables/part_d_analytical_ols.csv};
    \addlegendentry{$analytical$}

    \addplot [blue, mark=none] table [x=degree, col sep=comma] {../outputs/tables/part_d_ols_mse_vanilla_lr=0.01.csv};
    \addlegendentry{$vanilla$}

    \addplot [orange, mark=none] table [x=degree, col sep=comma] {../outputs/tables/part_d_ols_mse_momentum_lr=0.01.csv};
    \addlegendentry{$momentum$}

    \addplot [green!70!black, mark=none] table [x=degree, col sep=comma] {../outputs/tables/part_d_ols_mse_adagrad_lr=0.01.csv};
    \addlegendentry{$adagrad$}

    \addplot [red, mark=none] table [x=degree, col sep=comma] {../outputs/tables/part_d_ols_mse_rmsprop_lr=0.01.csv};
    \addlegendentry{$rmsprop$}

    \addplot [purple, mark=none] table [x=degree, col sep=comma] {../outputs/tables/part_d_ols_mse_adam_lr=0.01.csv};
    \addlegendentry{$adam$}
\end{axis}
\end{tikzpicture}
\caption{Test MSE of OLS regression vs. polynomial degree for analytical solution and iterative optimizers (vanilla GD, momentum, Adagrad, RMSProp, Adam, $\eta=0.01$), illustrating performance differences and convergence behavior.}
\label{fig:AGD_OLS}
\end{figure}

For Ridge regression with $\lambda = 0.01$, as shown in the Figure \ref{fig:AGD_RIDGE}, RMSProp and Adam show a better convergence at higher degrees, but still the analytical solution is the best one. 
 *wait for code update*

% File: ols_mse.tex
\begin{figure}[H]
\begin{tikzpicture}
\begin{axis}[
    width=8cm,
    height=6cm,
    xlabel={Polynomial degree},
    ylabel={MSE (Test)},
    title={RIDGE MSE vs Polynomial degree(lr=0.01, lambda=0.01)},
    legend pos=north east,
    grid=both,
    y tick label style={
        /pgf/number format/fixed,
        /pgf/number format/precision=2
    },
    ]
    \addplot [black, mark=none, dotted] table [x=degree, col sep=comma] {../outputs/tables/part_d_analytical_ridge.csv};
    \addlegendentry{$analytical$}

    \addplot [blue, mark=none] table [x=degree, col sep=comma] {../outputs/tables/part_d_ridge_mse_vanilla_lr=0.01.csv};
    \addlegendentry{$vanilla$}

    \addplot [orange, mark=none] table [x=degree, col sep=comma] {../outputs/tables/part_d_ridge_mse_momentum_lr=0.01.csv};
    \addlegendentry{$momentum$}

    \addplot [green!60!black, mark=none] table [x=degree, col sep=comma] {../outputs/tables/part_d_ridge_mse_adagrad_lr=0.01.csv};
    \addlegendentry{$adagrad$}

    \addplot [red, mark=none] table [x=degree, col sep=comma] {../outputs/tables/part_d_ridge_mse_rmsprop_lr=0.01.csv};
    \addlegendentry{$rmsprop$}

    \addplot [purple, mark=none] table [x=degree, col sep=comma] {../outputs/tables/part_d_ridge_mse_adam_lr=0.01.csv};
    \addlegendentry{$adam$}
\end{axis}
\end{tikzpicture}
\caption{\textbf{caption here}}
\label{fig:AGD_RIDGE}
\end{figure}


\subsection{Lasso resulsts}
\textbf{Have a figure for different lr with standard methond for lasso, and then a figure for the advanced methods, we can then compare this to the previeus subsection to see what the dif is}
% File: ols_mse.tex
\begin{figure}[H]
\begin{tikzpicture}
\begin{axis}[
    width=8cm,
    height=6cm,
    xlabel={Polynomial degree},
    ylabel={MSE},
    title={LASSO Regression With Different Learning Rates},
    legend pos=north east,
    grid=both,
    y tick label style={
        /pgf/number format/fixed,
        /pgf/number format/precision=2
    }
    ]
    \addplot [blue, mark=none] table [x=degree, col sep=comma] {../outputs/tables/part_e_lasso_mse_vanilla_lr=0.01.csv};
    \addlegendentry{$lr=0.01$}

    \addplot [orange, mark=none] table [x=degree, col sep=comma] {../outputs/tables/part_e_lasso_mse_vanilla_lr=0.001.csv};
    \addlegendentry{$lr=0.001$}

    \addplot [green!70!black, mark=none] table [x=degree, col sep=comma] {../outputs/tables/part_e_lasso_mse_vanilla_lr=0.0001.csv};
    \addlegendentry{$lr=0.0001$}
\end{axis}
\end{tikzpicture}
\caption{Test MSE of Lasso regression vs. polynomial degree using vanilla GD for learning rates $\eta=0.01$, $0.001$, and $0.0001$, showing the effect of step size on convergence and generalization.}
\label{fig:GD_LASSO_LR}
\end{figure}

% File: ols_mse.tex
\begin{figure}[H]
\begin{tikzpicture}
\begin{axis}[
    width=8cm,
    height=6cm,
    xlabel={Polynomial degree},
    ylabel={MSE},
    title={LASSO Regression With Advanced GD Methods},
    legend pos=north east,
    grid=both,
    y tick label style={
        /pgf/number format/fixed,
        /pgf/number format/precision=2
    },
    ]
    \addplot [blue, mark=none] table [x=degree, col sep=comma] {../outputs/tables/part_e_lasso_mse_vanilla_lr=0.01.csv};
    \addlegendentry{$vanilla$}

    \addplot [orange, mark=none] table [x=degree, col sep=comma] {../outputs/tables/part_e_lasso_mse_momentum_lr=0.01.csv};
    \addlegendentry{$momentum$}

    \addplot [green!60!black, mark=none] table [x=degree, col sep=comma] {../outputs/tables/part_e_lasso_mse_adagrad_lr=0.01.csv};
    \addlegendentry{$adagrad$}

    \addplot [red, mark=none] table [x=degree, col sep=comma] {../outputs/tables/part_e_lasso_mse_rmsprop_lr=0.01.csv};
    \addlegendentry{$rmsprop$}

    \addplot [purple, mark=none] table [x=degree, col sep=comma] {../outputs/tables/part_e_lasso_mse_adam_lr=0.01.csv};
    \addlegendentry{$adam$}
\end{axis}
\end{tikzpicture}
\caption{Test MSE of LASSO regression vs. polynomial degree for vanilla GD, momentum, Adagrad, RMSProp, and Adam ($\eta=0.01$, $\lambda=0.01$), illustrating the effect of model complexity on generalization.}
\label{fig:AGD_LASSO}
\end{figure}


\subsection{Stochastic Gradient Descent}
\textbf{Here we will discuss stachastic gradient descent with 3 figures of the 3 regressions methods but with sgd}
% File: ols_mse.tex
\begin{figure}[H]
\begin{tikzpicture}
\begin{axis}[
    width=8cm,
    height=6cm,
    xlabel={Polynomial degree},
    ylabel={MSE},
    title={OLS With Advanced SGD Methods},
    legend pos=north east,
    grid=both,
    y tick label style={
        /pgf/number format/fixed,
        /pgf/number format/precision=2
    }
    ]
    \addplot [blue, mark=none] table [x=degree, col sep=comma] {../outputs/tables/part_f_ols_mse_vanilla_lr=0.01_sgd.csv};
    \addlegendentry{$vanilla$}

    \addplot [orange, mark=none] table [x=degree, col sep=comma] {../outputs/tables/part_f_ols_mse_momentum_lr=0.01_sgd.csv};
    \addlegendentry{$momentum$}

    \addplot [green!70!black, mark=none] table [x=degree, col sep=comma] {../outputs/tables/part_f_ols_mse_adagrad_lr=0.01_sgd.csv};
    \addlegendentry{$adagrad$}

    \addplot [red, mark=none] table [x=degree, col sep=comma] {../outputs/tables/part_f_ols_mse_rmsprop_lr=0.01_sgd.csv};
    \addlegendentry{$rmsprop$}

    \addplot [purple, mark=none] table [x=degree, col sep=comma] {../outputs/tables/part_f_ols_mse_adam_lr=0.01_sgd.csv};
    \addlegendentry{$adam$}
\end{axis}
\end{tikzpicture}
\caption{Test MSE of OLS regression for different polynomial degrees using stachastic-gradient-based optimizers. The plot compares vanilla gradient descent, momentum, Adagrad, RMSProp, and Adam ($learning rate = 0.01$), highlighting their impact on convergence and generalization performance across increasing model complexity.}
\label{fig:AGD_OLS_SGD}
\end{figure}
% File: ols_mse.tex
\begin{figure}[H]
\begin{tikzpicture}
\begin{axis}[
    width=8cm,
    height=6cm,
    xlabel={Polynomial degree},
    ylabel={MSE},
    title={RIDGE Regression With Advanced SGD Methods},
    legend pos=north east,
    grid=both,
    y tick label style={
        /pgf/number format/fixed,
        /pgf/number format/precision=2
    },
    ]
    \addplot [blue, mark=none] table [x=degree, col sep=comma] {../outputs/tables/part_f_ridge_mse_vanilla_lr=0.01_sgd.csv};
    \addlegendentry{$vanilla$}

    \addplot [orange, mark=none] table [x=degree, col sep=comma] {../outputs/tables/part_f_ridge_mse_momentum_lr=0.01_sgd.csv};
    \addlegendentry{$momentum$}

    \addplot [green!60!black, mark=none] table [x=degree, col sep=comma] {../outputs/tables/part_f_ridge_mse_adagrad_lr=0.01_sgd.csv};
    \addlegendentry{$adagrad$}

    \addplot [red, mark=none] table [x=degree, col sep=comma] {../outputs/tables/part_f_ridge_mse_rmsprop_lr=0.01_sgd.csv};
    \addlegendentry{$rmsprop$}

    \addplot [purple, mark=none] table [x=degree, col sep=comma] {../outputs/tables/part_f_ridge_mse_adam_lr=0.01_sgd.csv};
    \addlegendentry{$adam$}
\end{axis}
\end{tikzpicture}
\caption{Comparison of Ridge regression test MSE using analytical and stachastic-gradient-based methods. The plot shows mean squared error across polynomial degrees for the analytical solution, vanilla gradient descent, momentum, Adagrad, RMSProp, and Adam ($learning rate = 0.01$), highlighting differences in convergence and generalization performance between exact and iterative approaches.}
\label{fig:AGD_RIDGE_SGD}
\end{figure}

% File: ols_mse.tex
\begin{figure}[H]
\begin{tikzpicture}
\begin{axis}[
    width=8cm,
    height=6cm,
    xlabel={Polynomial degree},
    ylabel={MSE (Test)},
    title={LASSO MSE vs Polynomial degree(lr=0.01, lambda=0.01)},
    legend pos=north east,
    grid=both,
    y tick label style={
        /pgf/number format/fixed,
        /pgf/number format/precision=2
    },
    ]
    \addplot [blue, mark=none] table [x=degree, col sep=comma] {../outputs/tables/part_f_lasso_mse_momentum_lr=0.01_sgd.csv};
    \addlegendentry{$momentum$}

    \addplot [orange, mark=none] table [x=degree, col sep=comma] {../outputs/tables/part_f_lasso_mse_adagrad_lr=0.01_sgd.csv};
    \addlegendentry{$adagrad$}

    \addplot [green!60!black, mark=none] table [x=degree, col sep=comma] {../outputs/tables/part_f_lasso_mse_rmsprop_lr=0.01_sgd.csv};
    \addlegendentry{$rmsprop$}

    \addplot [red, mark=none] table [x=degree, col sep=comma] {../outputs/tables/part_f_lasso_mse_adam_lr=0.01_sgd.csv};
    \addlegendentry{$adam$}
\end{axis}
\end{tikzpicture}
\caption{\textbf{caption here}}
\label{fig:AGD_LASSO_SGD}
\end{figure}


\subsection{bias variance trade off}
% File: ols_mse.tex
\begin{figure}[H]
\begin{tikzpicture}
\begin{axis}[
    width=8cm,
    height=6cm,
    xlabel={Polynomial degree},
    ylabel={MSE},
    title={Bias-Variance Tradeoff},
    legend pos=north east,
    grid=both,
    y tick label style={
        /pgf/number format/fixed,
        /pgf/number format/precision=2
    }
    ]
    \addplot [blue, mark=none] table [x=degree, col sep=comma] {../outputs/tables/part_g_mse_n=100.csv};
    \addlegendentry{$MSE$}

    \addplot [orange, mark=none] table [x=degree, col sep=comma] {../outputs/tables/part_g_bias2_n=100.csv};
    \addlegendentry{$Bias^2$}

    \addplot [green!70!black, mark=none] table [x=degree, col sep=comma] {../outputs/tables/part_g_variance_n=100.csv};
    \addlegendentry{$Variance$}
\end{axis}
\end{tikzpicture}
\caption{Decomposition of test MSE into bias and variance across polynomial degrees. The plot illustrates the bias–variance tradeoff: low-degree models exhibit high bias and low variance, while high-degree models show low bias and increased variance, with MSE reflecting the combined effect.}
\label{fig:Bias_Var}
\end{figure}

\subsection{Cross Validation}
% File: ols_mse.tex
\begin{figure}[H]
\begin{tikzpicture}
\begin{axis}[
    width=8cm,
    height=6cm,
    xlabel={Polynomial degree},
    ylabel={MSE},
    title={Cross Validation},
    legend pos=north west,
    grid=both,
    y tick label style={
        /pgf/number format/fixed,
        /pgf/number format/precision=2
    }
    ]
    \addplot [blue, mark=none] table [x=degree, col sep=comma] {../outputs/tables/part_h_cv_mse_ols.csv};
    \addlegendentry{$OLS$}

    \addplot [orange, mark=none] table [x=degree, col sep=comma] {../outputs/tables/part_h_cv_mse_ridge.csv};
    \addlegendentry{$RIDGE$}

    \addplot [green!70!black, mark=none] table [x=degree, col sep=comma] {../outputs/tables/part_h_cv_mse_lasso.csv};
    \addlegendentry{$LASSO$}
\end{axis}
\end{tikzpicture}
\caption{Cross-validation test MSE vs. polynomial degree for OLS, Ridge, and LASSO regression, showing how regularization in Ridge and LASSO mitigates overfitting compared to OLS.}
\label{fig:Cross_Val}
\end{figure}

\section{Discussion and Conclusion}
Summarize findings, bias-variance trade-off, comparisons.

\bibliographystyle{apsrev4-2}
\bibliography{References} % Your .bib file



\end{document}
